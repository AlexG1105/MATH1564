\documentclass{article}
\usepackage{blindtext}
\usepackage[left=3.5cm, right=3.5cm]{geometry}
\usepackage{amsmath}
\usepackage{amsfonts}
\usepackage{enumitem}
\title{\large{\vspace{-1.0cm}MATH-1564, K1, TA: Sam, Instructor: Nitzan, Sigal Shahaf \\ HW4 ; Alexander Guo}}
\date{}

\begin{document}

\maketitle

\vspace{-1.5cm}
\large

\begin{enumerate}

\item

\begin{enumerate}

\item $AB = \left(\begin{array}{ccc} 5 & 2 & 5 \\ 0 & -1 & 5 \end{array}\right)$
\item $BA =$ Undefined, inner dims don't match. $(2 \times \underline{3}) (\underline{2} \times 2)$
\item $D^{2} = \left(\begin{array}{ccc} 7 & -3 & 6 \\ -2 & 3 & -1 \\ 3 & -2 & 9 \end{array}\right)$ 
\item $B^{2} =$ Undefined, inner dims don't match.  $(2 \times \underline{3}) (\underline{2} \times 3)$
\item $DC = \left(\begin{array}{cc} 7 & -1 \\ 0 & 5 \\ 1 & 2 \end{array} \right)$
\item $CB = \left(\begin{array}{ccc} 5 & 2 & 5 \\ 5 & 1 & 10 \\ 0 & -1 & 5 \end{array} \right)$
\item $BC = \left(\begin{array}{cc} 7 & -1 \\ 7 & 4 \end{array} \right)$
\item $FE = \left(\begin{array}{c} 2 \end{array} \right)$
\item $EF = \left(\begin{array}{ccc} 1 & 2 & 3 \\ 2 & 4 & 6 \\ -1 & -2 & -3 \end{array} \right)$
\item $CE =$ Undefined, inner dims don't match.  $(3 \times \underline{2}) (\underline{3} \times 1)$
\item $EC =$ Undefined, inner dims don't match.  $(3 \times \underline{1}) (\underline{3} \times 2)$

\end{enumerate}

\item

\begin{enumerate}

\item

\begin{enumerate}

\item A is invertible. $A^{-1} =  \left(\begin{array}{cc} 0.2 & 0.4 \\ 0.4 & -0.2 \end{array} \right)$
\item B is not invertible. Its REF is $\left(\begin{array}{cc} 1 & -3 \\ 0 & 0 \end{array} \right)$
\item C is invertible. $C^{-1} =  \left(\begin{array}{cc} 2 & -3 \\ -1 & 2 \end{array} \right)$
\item D is invertible. $D^{-1} =  \left(\begin{array}{ccc} 0.5 & -0.5 & 0.5 \\ 0.5 & 0.5 & -0.5 \\ -0.5 & 0.5 & 0.5 \end{array} \right)$
\item E is not invertible. Its REF is $\left(\begin{array}{ccc} 1 & 0 & \frac{2}{3} \\ 0 & 1 & \frac{10}{3} \\ 0 & 0 & 0\end{array}\right)$

\end {enumerate}

\item 
$
A \left(\begin{array}{c} x \\ y \end{array} \right) = \left(\begin{array}{c} 1 \\ -2 \end{array} \right)
\rightarrow
A^{-1} A \left(\begin{array}{c} x \\ y \end{array} \right) = A^{-1} \left(\begin{array}{c} 1 \\ -2 \end{array} \right)
\rightarrow
\left(\begin{array}{c} x \\ y \end{array} \right) = \left(\begin{array}{cc} 0.2 & 0.4 \\ 0.4 & -0.2 \end{array} \right) \left(\begin{array}{c} 1 \\ -2 \end{array} \right)
\rightarrow
\left(\begin{array}{c} x \\ y \end{array} \right) =  \left(\begin{array}{c} -0.6 \\ 0.8 \end{array} \right)
$

\item
$
B \left(\begin{array}{c} x \\ y \end{array} \right) = \left(\begin{array}{c} 1 \\ -2 \end{array} \right)
\rightarrow
\left(\begin{array}{cc} 1 & -3 \\ -2 & 6 \end{array} \right) \left(\begin{array}{c} x \\ y \end{array} \right) = \left(\begin{array}{c} 1 \\ -2 \end{array} \right)
\rightarrow
\left(\begin{array}{c} x - 3y \\ -2x + 6y \end{array} \right) = \left(\begin{array}{c} 1 \\ -2 \end{array} \right)
\xrightarrow{R_2 + 2R_1} 
\left(\begin{array}{c} x - 3y \\ 0 \end{array} \right) =  \left(\begin{array}{c} 1 \\ 0 \end{array} \right)
$
So: 
$
\left(\begin{array}{c} x \\ y \end{array} \right) =  \left(\begin{array}{c} 1\\ 0 \end{array} \right) + s \left(\begin{array}{c} 3 \\ 1 \end{array} \right)
$

\item
$
DG = E
\rightarrow
D^{-1}DG = D^{-1}E
\rightarrow
G = \\
\left(\begin{array}{ccc} 0.5 & -0.5 & 0.5 \\ 0.5 & 0.5 & -0.5 \\ -0.5 & 0.5 & 0.5 \end{array} \right)
\left(\begin{array}{ccc} 3 & 0 & 2 \\ 2 & -1 & -2 \\ -1 & 2 & 6 \end{array} \right)
\rightarrow
G = \left(\begin{array}{ccc} 0 & 1.5 & 5 \\ 3 & -1.5 & -3 \\ -1 & 0.5 & 1 \end{array} \right)  
$
\end{enumerate}

\item

\begin{enumerate}

\item \textbf{False}. If  $A \in M_{n}(\mathbb{R})$, fix $n = 2$, let us say that $A =  \left(\begin{array}{cc} 0 & 1 \\ 0 & 0 \end{array} \right)$. $A^{2}$ = 0, but $A \neq 0$, thus the statement is false.
\item \textbf{True}. Let $A,B \in M_{n}(\mathbb{R})$. Rewrite $AB^{2} = B^{2}A$ as $A(BB) = (BB)A$. Apply associative property on matrices: $(AB)B = B(BA)$. Since we are given that $AB = BA$, substitute in equation: $(BA)B = B(AB)$. Apply associative property again and get $B(AB) = B(AB)$. We therefore conclude that the terms are equal and that $AB^2 = B^2A$ as long as $AB = BA$.
\item \textbf{False}. If $A,B,C \in M_{n}(\mathbb{R})$, fix $n = 2$, let us set
$A = \left(\begin{array}{cc} 0 & 0 \\ 1 & 1 \end{array} \right)$,
$B = \left(\begin{array}{cc} 1 & 1 \\ 1 & 1 \end{array} \right)$,
$C = \left(\begin{array}{cc} 0 & 0 \\ 2 & 0 \end{array} \right)$.
$AB = \left(\begin{array}{cc} 0 & 0 \\ 2 & 2 \end{array} \right)$, and $CB = \left(\begin{array}{cc} 0 & 0 \\ 2 & 2 \end{array} \right)$. However, $A \neq C$, so clearly the statement is false.

\end{enumerate}

\item

\begin{enumerate}

\item \textbf{True}. If A,B are invertible square matrices, then we need to prove that $B^{-1}A^{-1}$ is the inverse of AB. Let us denote $B^{-1}A^{-1}$ as C. We want to show that $(AB)C = C(AB) = I$ since it is the definition of inverse matrices. Thus, if we plug in for C: $(AB)(B^{-1}A^{-1})$, by the associative property for matrices, $A(BB^{-1})A^{-1} = AIA^{-1} = AA^{-1} = I$. Indeed, $AB$ is invertible, and its inverse is equal to $B^{-1}A^{-1}$. This also works when we plug C in to the left side: $(B^{-1}A^{-1})(AB) = B^{-1}IB = I$

\item \textbf{True}. If AB is invertible, then there exists a C so that $C(AB) = I$. Because matrices are multiplicatively associative, then we get $(CA)B = I$. We claim that B is invertable if $XB = I$. Denote $CA$ as $X$. Therefore, B is indeed invertible. On the other hand, if AB is invertible there exists a C so that $(AB)C = I$. Given the multiplicative association property of matrices, we get $A(BC)$. We claim that A is invertible if $AX = I$. Denote $CA$ as $X$. Therefore, A is indeed invertible.

\item \textbf{False}. If $A,B \in M_{n}(\mathbb{R})$, fix $n = 2$, let us say that $A = \left(\begin{array}{cc} 1 & 1 \\ 0 & 0 \end{array} \right)$,
$B = \left(\begin{array}{cc} 0 & 1 \\ 3 & 4 \end{array}\right)$.$A + B = \left(\begin{array}{cc} 1 & 2 \\ 3 & 4 \end{array} \right)$, and its inverse is
$\left(\begin{array}{cc} -2 & 1 \\ 1.5 & -0.5 \end{array} \right)$, so it is indeed invertible. However, A has no inverse, so the statement is false.

\item \textbf{False}. If $A,B \in M_{n}(\mathbb{R})$, fix $n = 2$, let us say that $A = \left(\begin{array}{cc} 1 & 2 \\ 3 & 4 \end{array} \right)$,
$B = \left(\begin{array}{cc} 0 & -1 \\ -3 & -4 \end{array}\right)$.$A^{-1} = \left(\begin{array}{cc} -2 & 1 \\ 1.5 & -0.5 \end{array}\right)$,
$B^{-1} = \left(\begin{array}{cc} \frac{4}{3} & -\frac{1}{3} \\ -1 & 0 \end{array}\right)$ so A,B are both invertible.
$A + B = \left(\begin{array}{cc} 1 & 1 \\ 0 & 0 \end{array} \right)$, but the resulting matrix has no inverse. Therefore, the statement is false.

\item \textbf{True}. If $AB$ is invertible, then from (b), we know that A and B are both invertible. Then, from question (a), we know that the multiplication of two invertible matrices are also invertible. Hence, $BA$, is invertible, proving our claim to be true.

\item \textbf{True}. If $A^3$ is invertible, then there exists a B so that $A^3B = I$. We can rearrange this expression to $(AAA)B = I$. With the associative property of matrix products, we have $A(AAB) = I$. We claim that A is invertible if there exists some X so that $AX = I$. Indeed, denote $AAB$ as $X$, so $AX = I$, therefore A is invertible and the statement is true.

\end{enumerate}

\item

\begin{enumerate}

\item Let $A,B \in M_n(\mathbb{R})$, then set $A_{ij} = 0$ for all $i \neq j$. So, $A = \left(\begin{array}{ccc} a_{11} & 0 & 0 \\ 0 & ... & 0 \\ 0 & 0 & a_{nn} \end{array}\right)$ 
and $B = \left(\begin{array}{ccc} b_{11} & 0 & 0 \\ 0 & ... & 0 \\ 0 & 0 & b_{nn} \end{array}\right)$. By regular matrix addition, each respective coordinate is added $a_{ij} + b_{ij}$, and
$A + B = \left(\begin{array}{ccc} a_{11} + b_{11} & 0 & 0 \\ 0 & ... & 0 \\ 0 & 0 & a_{nn} + b_{nn} \end{array}\right)$. Therefore, we conclude that $A + B$ is diagonal.
Similarly, by regular matrix multiplication, $AB = \left(\begin{array}{ccc} a_{11} b_{11} & 0 & 0 \\ 0 & ... & 0 \\ 0 & 0 & a_{nn}b_{nn} \end{array}\right)$, so we also conclude that $AB$ is diagonal.

\item Let $A,B,C,D \in M_n(\mathbb{R})$. Denote $AB$ as $C$, then in sum notation for matrix multiplication, $c_{ij} = \sum\limits_{k = 1}^n a_{ik}b_{kj}$, where $1 \leq i \leq n, 1 \leq j \leq n$. Since we are only interested in the trace of matrix C, which is given to us as $tr(C) = \sum\limits_{i = 1}^n (c)_{ii}$, we substitute in our multiplication equation and get
$tr(C) = \sum\limits_{i = 1}^n \sum\limits_{k = 1}^n a_{ik}b_{ki}$. On the other hand, denote $BA$ as $D$, then in sum notation for matrix multiplication, $d_{ij} =  \sum\limits_{k = 1}^n b_{ik}a_{kj}$, where $1 \leq i \leq n, 1 \leq j \leq n$. Since we are only interested in the trace of matrix D, which is given to us as $tr(D) = \sum\limits_{i = 1}^n (d)_{ii}$, we substitute in our multiplication equation and get $tr(D) = \sum\limits_{i = 1}^n \sum\limits_{k = 1}^n b_{ik}a_{ki}$. Then, we reverse the sigmoids, and get $tr(D) = \sum\limits_{k = 1}^n \sum\limits_{i = 1}^n a_{ki}b_{ik}$. Because i and k take on the same values, they are interchangable, and thus tr(C) and tr(D) are equivalent. Therefore, $tr(AB) = tr(BA)$.

\item The $ij$ entry of $AB$ can be written as $AB_{ij} = \sum\limits_{k = 1}^n a_{ik}b_{kj}$. If we transpose a matrix, we switch the rows with the columns, so\\ $(AB_{ij})^T = AB_{ji} = \sum\limits_{k = 1}^n a_{jk}b_{ki}$.\\ On the other hand, \\$(B^TA^T)_{ij} = \sum\limits_{k = 1}^n (B^T)_{ik}(A^T)_{kj} = \sum\limits_{k = 1}^n b_{ki}a_{jk}$. \\Since real numbers are multiplicatively communitive, these two outcomes are the same and the matrices are therefore equal. Thus, $(AB)^T = B^TA^T$.

\end{enumerate}

\end{enumerate}

\end{document}