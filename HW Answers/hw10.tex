\documentclass{article}
\usepackage{blindtext}
\usepackage[left=2.5cm, right=2.5cm]{geometry}
\usepackage{amsmath}
\usepackage{amsfonts}
\usepackage{enumitem}
\usepackage{systeme}
\title{\large{\vspace{-1.0cm}MATH-1564, K1, TA: Sam, Instructor: Nitzan, Sigal Shahaf \\ HW10 ; Alexander Guo}}
\date{}

\begin{document}

\maketitle

\vspace{-1.5cm}


\begin{enumerate}

\item

\begin{enumerate}

\item \textbf{Is a linear transformation.}  We first want to prove that this satisfies the summation condition. For $v_1,v_2 \in V$ where $v_1 = \left(\begin{array}{c} x_1 \\ y_1 \\ z_1 \end{array}\right), v_2 = \left(\begin{array}{c} x_2 \\ y_2 \\ z_2 \end{array}\right)$,\\ $T(v_1 + v_2) = \left(\begin{array}{cc} (x_1 + x_2) + (y_1 + y_2) & (y_1 + y_2) - 2 (z_1 + z_2) \\ 3(x_1 + x_2) + (z_1 + z_2) & 0 \end{array}\right)$. Through properties of real numbers and matrices, we arrange the statement to $\left(\begin{array}{cc} x_1 + y_1 & y_1 - 2z_1 \\ 3x_1 + z_1 & 0 \end{array}\right) + \left(\begin{array}{cc} x_2 + y_2 & y_2 - 2z_2 \\ 3x_2 + z_2 & 0 \end{array}\right)$. This is equivalent to $T (v_1) + T (v_2)$ and therefore satisfies summation. Next we want to prove the multiplication condition. For the same $v_1$ and scalar $a \in \mathbb{R}$, we show that $T(av_1) = \left(\begin{array}{cc} a(x_1 + y_1) & a(y_1 - 2z_1) \\ a(3x_1 + z_1) & 0 \end{array}\right)$. With the definition of matrix multiplication by scalar, this is equivalent to \\ $a \left(\begin{array}{cc} x_1 + y_1 & y_1 - 2z_1 \\ 3x_1 + z_1 & 0 \end{array}\right) = a T(v_1)$ which satisfies multiplication by scalar.

\item \textbf{Is a linear transformation.} To prove summation condition, for $v_1,v_2 \in V$ and $a,b \in \mathbb{R}$ where $v_1 = a_1 x^2 + a_2 x + a_3, v_2 = b_1 x^2 + b_2 x + b_3$, we want to show that $T(v_1 + v_2) = T(v_1) + T(v_2)$. $T(v_1 + v_2) = \left(\begin{array}{c} 4(a_1 + b_1) + 2(a_2 + b_2) + (a_3 + b_3) \\ 4(a_1 + b_1) + (a_2 + b_2) \\ 2(a_1 + b_1)\end{array}\right) = \left(\begin{array}{c} 4a_1 + 2a_2 + a_3 \\ 4a_1 + a_2 \\ 2a_1 \end{array}\right) + \left(\begin{array}{c} 4b_1 + 2b_2 + b_3 \\ 4b_1 + b_2 \\ 2b_1 \end{array}\right) = T(v_1) + T(v_2)$. Next we want to prove multiplication property. Using $v_1$ from before and scalar $s \in \mathbb{R}$, $T(sv_1) = \left(\begin{array}{c} s (4a_1 + 2a_2 + a_3) \\ s (4a_1 + a_2) \\ s(2a_1) \end{array}\right) = s T(v_1)$. This satisfies the multiplication property.

\item \textbf{Not a linear transformation.} Say our matrix in v is $\left(\begin{array}{cc} 1 & 1 \\ 0 & 0 \end{array}\right)$. $T(2v) = \left(\begin{array}{cc} 4 & 4 \\ 0 & 0 \end{array}\right)$. But this is equal to $4v$, so multiplication property is incorrect.

\item \textbf{Is a linear transformation.} To prove summation condition, for $v_1,v_2 \in V$ and $a,b \in \mathbb{R}$ where  $v_1 = a_1 x^2 + a_2 x + a_3, v_2 = b_1 x^2 + b_2 x + b_3$, we want to show that $T(v_1 + v_2) = T(v_1) + T(v_2)$. $T(v_1 + v_2) $ evaluates to $\int_{0}^{1} (a_1 + b_1)x^2 + (a_2+b_2)x + (a_3 + b_3) dx$. Since we can split integrals, this is also equivalent to $\int_{0}^{1} a_1x^2 + a_2x + a_3 dx + \int_{0}^{1} b_1x^2 + b_2x + b_3 dx = T(v_1) + T(v_2)$. Thus, summation is satisfied. Then, to prove the multiplication property, using $v_1$ and a scalar $s \in \mathbb{R}$, $T(sv_1) = \int_{0}^{1} s(a_1x^2 + a_2x + a_3) dx = s \int_{0}^{1} a_1x^2 + a_2x + a_3 dx = sT(v_1)$. Therefore, this satisfies the multiplication property.

\item \textbf{Is a linear transformation.} To prove summation condition, say $A_1, A_2$ are ambiguous matrices and $B$ is the fixed matrix. $T(A_1 + A_2) = (A_1 + A_2) B = (A_1 B) + (A_2 B) = T(A_1) + T(A_2)$. Thus it satisfies the summation condition. Next, to show that it satisfies multiplication, for some scalar $s \in \mathbb{R}$, $T(sA_1) = (sA_1)B = s(A_1 B) = sT(A_1)$. Thus, they are the equivalent and multiplication property is satisfied.

\item \textbf{Not a linear transformation.} $T\left(\begin{array}{cc} 1 & 1 \\ 1 & 1 \end{array}\right) = \left(\begin{array}{cc} 1 & 7 \\ -8 & 2 \end{array}\right)$. If we scale this matrix by 2, $T(2A) = \left(\begin{array}{cc} 1 & 12 \\ -8 & 4 \end{array}\right)$. However, $2 T(A) = \left(\begin{array}{cc} 2 & 14 \\ -16 & 4 \end{array}\right)$. Thus, $T(2A) \neq 2T(A)$ and we do not have a linear transformation.

\end{enumerate}

\item

\begin{enumerate}

\item 

\begin{enumerate}

\item The basis for the kernel is $\left(\begin{array}{c} -1 \\ 1 \\ -1 \end{array}\right)$. The basis for the image is $\left(\left(\begin{array}{c} 1 \\ 0 \\ 0 \\ 0 \end{array}\right), \left(\begin{array}{c} 0 \\ 1 \\ 0 \\ 0 \end{array}\right)\right)$.

\item The dimension of the kernel is 1, and the dimension of the image is 2.

\item The transformation is not onto, since according the theorem from class, T is onto if and only if the image of T is equal to W. In this case, $\mathbb{R}^4$ is not spanned by the basis for the image since its dimension is only 2.

\item The transformation is not 1-1. From the theorem in class, T is 1-1 if and only if its kernel is equal to 0, which in this case, it is not.

\end{enumerate}

\item

\begin{enumerate}

\item The basis for the kernel is $\left(\left(\begin{array}{cc} 2 & 3 \\ 0 & 0 \end{array}\right),\left(\begin{array}{cc} 0 & 0 \\ 2 & 3 \end{array}\right)\right)$. The basis for the image is $\left(\left(\begin{array}{c} 1 \\ 0 \end{array}\right), \left(\begin{array}{c} 0 \\ 1 \end{array}\right)\right)$.

\item The dimension of the kernel is 2, and the dimension of the image is 2.

\item S is onto if and only if W equals image. This holds true, since the basis for the image spans all of $\mathbb{R}^2$. Therefore, the transformation is onto.

\item S is 1-1 if and only if the kernel equals 0. Since the dimension of the kernel is not 0, it is not equal to 0, and thus the transformation is not 1-1.

\end{enumerate}

\item

\begin{enumerate}

\item The basis for the kernel is $(-\frac{3}{7}x^3 + x^2,1)$, and the basis for the image is $\left(\left(\begin{array}{c} 1 \\ 0 \end{array}\right), \left(\begin{array}{c} 0 \\ 1 \end{array}\right)\right)$

\item The dimension of the kernel is 2, and the dimension of the image is 2.

\item L is onto if and only if W equals image. This holds true, since the basis for the image spans all of $\mathbb{R}^2$. Therefore, the transformation is onto.

\item L is 1-1 if and only if the kernel equals 0. Since the dimension of the kernel is not 0, it is not equal to 0, and thus the transformation is not 1-1.

\end{enumerate}

\item

\begin{enumerate}

\item The basis for the kernel is the empty set, and the basis for the image is $(1,x,x^2,x^3)$.

\item The dimension of the kernel is 0, and the dimension of the image is 4.

\item Onto, because W equals image. In other words, the vector space $\mathbb{R}_3[x]$ is spanned by the basis for the image.

\item 1-1, because the kernel equals 0, since its dimension is 0.

\end{enumerate}

\end{enumerate}

\item

\begin{enumerate}

\item This statement is false (it has to be 1-1 in order for it to be true). Let us say T is the linear transformation that maps vectors to the real number 0. The vectors $\left(\begin{array}{c} 1 \\ 0 \end{array}\right), \left(\begin{array}{c} 0 \\ 1 \end{array}\right)$ are linearly independent, but the linear transformation of the two vectors are both equal to 0, meaning that they are not linearly independent.

\item True. Let us consider for all scalars $a \in \mathbb{R}$, $a_1v_1 + ... + a_nv_n = 0$. Since the kernel maps over to the zero of W, we have: $$T(0_v) = T(a_1 v_1 + ... + a_n v_n)$$ $$0_w = T(a_1v_1) + ... + T(a_nv_n)$$ $$0_w = a_1T(v_1) + ... + a_nT(v_n)$$ Thus, $a_1,...,a_n$ is equal to zero since we are given that $Tv_1,...,Tv_n$ is linearly independent. That would also imply in our original equation that $v_1,...,v_n$ is linearly independent in V.

\item This statement is false. Let us say $\left(\begin{array}{c} 1 \\ 0 \end{array}\right),\left(\begin{array}{c} 0 \\ 1 \end{array}\right)$ spans the space $\mathbb{R}^2$. Now, let us define T as the linear transformation $\mathbb{R}^2 \rightarrow \mathbb{R}^2$ which maps the vector $\left(\begin{array}{c} a \\ b \end{array}\right) \rightarrow \left(\begin{array}{c} a \\ 0 \end{array}\right)$. Thus, the vectors we gave that span $\mathbb{R}^2$ now become $\left(\begin{array}{c} 1 \\ 0 \end{array}\right), \left(\begin{array}{c} 0 \\ 0 \end{array}\right)$, but they clearly don't span W.

\item True. Since $Tv_1,...,Tv_n$ spans W, we know that the image of V is equal to W. Therefore, from a theorem proved in class, if image of V equals W, then T is onto. Next, since $Tv_1,...,Tv_n$ is a spanning set in W, we know that for some $a \in \mathbb{R}, w \in W$, $w = a_1Tv_1 + ... + a_nTv_n$. If we recall, since T is onto, it means that every w has at least one preimage in V such that $Tv = w$. Thus, it is equivalent to say that $Tv = T(a_1v_1 + ... + a_nv_n)$, then $v = a_1v_1 + ... + a_nv_n$ for all $v \in V$, so it thus suffices to say that $v_1,...,v_n$ spans V.

\item To prove that T(U) is a subspace of W, we first want to show that it is not empty. We know that it is not empty because U, which is a subspace of V, must contain the $0_v$ which maps over to $0_w$ by definition. Thus T(U) must contain $0_w$, so it is not empty. Next, we want to show that it is closed to addition. So, say that $u_1,u_2$ are in T(U) such that $T(v_1) = u_1$ and $T(v_2) = u_2$. We then claim that $v_1 + v_2$ is a preimage of $u_1 + u_2$. Then, $T(v_1 + v_2) = Tv_1 + Tv_2 = u_1 + u_2$. Thus, $u_1 + u_2 \in T(U)$. Finally, we want to prove that it is closed to multiplication by scalar. This is very similar, we simply take scalar $a \in \mathbb{R}$ and $u_1 \in T(U)$ such that $T(v_1) = u_1$. We then claim that $av_1$ is a preimage of $au_1$. So, $T(av_1) = aT(v_1) = au_1$, and $au_1 \in T(U)$. Therefore, all three properties are satisfied and T(U) is a subspace of W.

\item Let us say T:$\mathbb{R}^2 \rightarrow \mathbb{R}^2$ is a linear transformation given by the following: $\left(\begin{array}{c} a \\ b \end{array}\right) \rightarrow \left(\begin{array}{c} 0 \\ 0 \end{array}\right)$. So, a let's chose a single vector subset of V given by $\left(\begin{array}{c} 1 \\ 0 \end{array}\right)$. When we apply the linear transformation, it becomes $\left(\begin{array}{c} 0 \\ 0 \end{array}\right)$ which is a subspace of $\mathbb{R}^2$. However, the original vector is not a subspace of $\mathbb{R}^2$.

\end{enumerate}

\item \textbf{and} 5.

\begin{enumerate}

\item \textbf{False.} From the corollary, T is 1-1 if dimV $\leq$ dimW. The dimension of $M_2(\mathbb{R})$ is 4 and the dimension of $\mathbb{R}^3$ is 3, so it does not satisfy this relationship.

\begin{enumerate}

\item Stays false.

\item Adding the condition that T is onto will make it true. Since in this question V has dimension of 4 and W has dimension of 3, there will definitely exist some T that will be onto. This is given by the corollary that if T is onto, then dimV $\geq$ dimW.

\end{enumerate}

\item \textbf{True.} From the corollary, T is 1-1 if dimV $\leq$ dimW. The dimension of $\mathbb{R}^3$ is 3 and the dimension of $M_2(\mathbb{R})$ is 4, so this relationship is satisfied.

\item \textbf{True.} An example is mapping any matrix to the zero matrix: $\left(\begin{array}{cc} a & b \\ c & d \end{array}\right) \rightarrow \left(\begin{array}{cc} 0 & 0 \\ 0 & 0 \end{array}\right)$. Clearly, this relationship is not 1-1 since more than one matrix maps to the 0 matrix, and this relationship is not onto, because matrices cannot be mapped to anything other than the 0 matrix.

\item \textbf{False.} From the corollary, if dimV $=$ dimW, and T is 1-1, then it must be onto. In this scenario, the dimension of $M_2(\mathbb{R})$ is 4, and the dimension of $\mathbb{R}_3[x]$ is 4. Since the dimensions are equal, and we are given that it is 1-1, it must be onto. However there is a contradiction because this question asks that it is not onto.

\begin{enumerate}

\item Stays false.

\item Adding the condition that T is onto makes this statement true. This is because V has a dimension of 4, and W has a dimension of 3. We know from the corollary if dimV $>$ dimW, then there must be a possible T that is onto.

\end{enumerate}

\item \textbf{True.} From the corollaries, T is onto if dimV $\geq$ dimW, and T is 1-1 if dimV $\leq$ dimW. Then the equality holds that if dimV $>$ dimW, T is onto and not 1-1. In this scenario, the dimension of V is 4, and the dimension of W is 3. Then there are indeed many examples for T that are onto and not 1-1.

\item \textbf{False.} T is onto if and only if the image of V is equal to W. In this scenario, the dimension of V is equal to 3, while the dimension of W is equal to 4. Clearly, the image of V cannot equal to W because its dimension is less than that of W. 

\begin{enumerate}

\item Since in this case, dimV $<$ dimW, we know that T could definitely be 1-1.

\item Stays false.

\end{enumerate}

\item \textbf{True.} An example is mapping the matrix to the zero vector: $\left(\begin{array}{cc} a & b \\ c & d \end{array}\right) \rightarrow \left(\begin{array}{c} 0 \\ 0 \\ 0\end{array}\right)$. This is not onto because nothing outside of zero is mapped to, and this is not 1-1 since all matrices are mapped to the 0 vector.

\end{enumerate}

\end{enumerate}

\end{document}